\section{Conclusions}

\frame
{
	\frametitle{Points forts}
	\begin{itemize}
		\item Norme compl\`ete et \'evolutive :
		\begin{itemize}
			\item<2-> Pas limit\'ee \`a la radiologie (oncologie, h\'ematologie, dermatologie, ophtalmologie, cardiologie, ORL,\ldots).
			\item<3-> Annotations sur les images, compression.
		\end{itemize}
		\item<4-> Libert\'e du choix d'achat d'\'equipement car ind\'ependance du fournisseur.
		\item<5-> Distribution d'images possible : support des protocoles de communication standards, TCP/IP notamment.
	\end{itemize}
}

\frame
{
	\frametitle{Points faibles}
	\begin{itemize}
		\item Complexe :
		\begin{itemize}
			\item<2-> Difficile \`a comprendre.
			\item<3-> Informatique de niche.
		\end{itemize}
		\item<4-> N\'ecessit\'e de patience : int\'egration progressive par les fournisseurs.
		\item<5-> P\'erim\`etre limit\'e : traite de la connectivit\'e, mais pas des fonctionnalit\'es des logiciels.
		\item<6-> Besoin d'un niveau de conformit\'e.
	\end{itemize}
}

\frame
{
	\frametitle{DICOM Conformance}
	\begin{itemize}
		\item La norme pr\'evoit un document "DICOM Conformance Statement" dont le plan et la structure sont pr\'ed\'efinis.
		\item<2-> Par ce document, le fournisseur pr\'ecise le niveau de conformit\'e de son \'equipement \`a la norme DICOM.
		\begin{itemize}
			\item<3-> Applicable sur chaque mod\`ele, chaque version.
			\item<4-> Le document suit un plan pr\'evu dans le standard.
			\item<5-> Liste des SOP Class support\'ees et des r\^oles assur\'es (SCU, SCP).
		\end{itemize}
	\end{itemize}
}

\frame
{
	\frametitle{Pi\`eges et difficult\'es}
	\begin{itemize}
		\item Divergences/erreurs d'interpr\'etation de la norme.
		\item<2-> DICOM autorise le stockage d'informations sp\'ecifiques.
		\begin{itemize}
			\item<3-> Risque: faire du propri\'etaire sous le label DICOM
		\end{itemize}
		\item<4-> DICOM propose diff\'erentes alternatives pour d\'ecrire l'information.
		\begin{itemize}
			\item<5-> Annotations: 3 moyens de les transmettre.
		\end{itemize}
		\item<6-> Manque d'information disponible \`a l'installation sur l'activation des services DICOM.
	\end{itemize}
}

\frame
{
	\frametitle{Quelques contre-v\'erit\'es}
	\begin{itemize}
		\item "Je n'arrive plus \`a envoyer mes images. C'est \`a cause de DICOM !"
		\begin{description}
			\item<2->[Faux] DICOM utilise le r\'eseau, qui peut avoir ses d\'efaillances.
		\end{description}
		\item<3-> "DICOM d\'egrade la qualit\'e de mes images."
		\begin{description}
			\item<4->[Faux] DICOM n'invente rien : il repose sur des formats d'image qui peuvent ou non \^etres compress\'es.
		\end{description}
		\item<5-> "Ne vous inqui\'etez pas, je supporte enti\`erement DICOM"
		\begin{description}
			\item<6->[Faux] Remarque bien pr\'esomptueuse\ldots
			Possible, mais est-ce r\'ealiste ?
		\end{description}
	\end{itemize}
}

\frame
{
	\frametitle{Synth\`ese}
	\begin{itemize}
		\item<1-> Norme incontournable.
		\item<2-> Tr\`es largement et rapidement adopt\'ee par la majorit\'e des acteurs.
		\item<3-> Plus large couverture que l'imagerie radiologique
		\begin{itemize}
			\item<4-> Rapport structur\'e.
			\item<5-> Ouverture \`a toutes les imageries.
		\end{itemize}
		\item<6-> Norme \'evolutive.
	\end{itemize}
}
