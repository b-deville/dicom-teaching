\section{Principes de DICOM}

	\subsection{Objectifs de DICOM}
	
	\frame
	{
		\frametitle{Buts globaux}
		\begin{itemize}
			\item<1-> Trouver un langage commun pour l'\'echange (images et donn\'ees pertinentes) entre \'equipements d'imagerie : mettre en place un standard.
			\item<2-> Pousser les vendeurs \`a supporter ce langage commun.
			\item<3-> Standardiser :
			\begin{itemize}
				\item<4-> le stockage (i.e. format de fichier) ;
				\item<5-> et la communication des donn\'es (i.e. protocoles de communication).
			\end{itemize}
		\end{itemize}
	}
	
	\frame
	{
		\frametitle{Buts pr\'ecis}
		
		Il faut que lors de l'installation d'une nouvelle modalit\'e, le DICOM permette, sans changement d'un quelconque composant logiciel :
		\begin{itemize}
			\item<1-> l'interrogation du PACS ;
			\item<2-> la r\'ecup\'eration des images cr\'e\'ees par d'autres syst\`emes ;
			\item<3-> l'affichage des images ;
			\item<4-> et la production d'images lisibles par les syst\`emes d'autres constructeurs.
		\end{itemize}
	}

	\subsection{Fondements th\'eoriques}

	\frame
	{
		\frametitle{Monde r\'eel}
		Au cours d'un suivi m\'edical, un patient se voit prescrire des examens radiologiques par son m\'edecin.
		
		Sch\'ematisation de la proc\'edure :
		
		\includegraphics[width=\linewidth]{./figures/scenario.png}
		
		DICOM d\'ecrit ces donn\'ees et relations.
		
		La pr\'ecision du contenu et des liens d\'epend des outils et des utilisateurs (e.g. RIS et PACS).
	}

	\frame
	{
		\frametitle{Traduire le r\'eel en num\'erique}
		
		Un objet DICOM combine donc :
		\begin{itemize}
			\item<2-> des donn\'ees, ou informations (e.g. nom du patient, donn\'ees de l'image,\ldots) ;
			\item<3-> et services, ou fonctions (e.g. sauvegarder, imprimer,\ldots).
		\end{itemize}
		
		Le traitement DICOM d'une information consiste alors \`a regrouper :
		\begin{itemize}
			\item<4-> l'\emph{Information Objet}, contenant les donn\'ees de l'objet, respectant une \emph{Information Object Definition} (ou \emph{IOD}) ;
			\item<5-> et une fonction sp\'ecifique, ou \emph{Service}, d\'efini par un \emph{DICOM Message Service Element}, ou \emph{DIMSE}.
		\end{itemize}
	}
	
	\frame
	{
		\frametitle{SOP Class UID}

		\begin{itemize}
			\item La combinaison Information Objet + Service est :
			\begin{itemize}
				\item<2-> appel\'ee \emph{Service/Object Pair} ou \emph{SOP} ;
				\item<3-> l'\'el\'ement principal de la conformit\'e au standard ;
				\item<4-> identifi\'ee par un identifiant unique nomm\'e \emph{SOP Class UID}.
			\end{itemize}
		
			\item<5-> Standard DICOM = annuaire de SOP.\\
			SOP Class UID = num\'ero unique pour trouver \`a quelle paire correspond un objet DICOM.\\
			Analogies : num\'ero AVS, adresse IP,\ldots

			\item<6-> Exemples de SOP Class UID :
			\begin{description}
				\item<7->[$1.2.840.10008.5.1.4.1.1.1$] CR Image Store (enregistrer un CR) ;
				\item<8->[$1.2.840.10008.5.1.4.1.1.2$] CT Image Store (enregistrer un CT).
			\end{description}
		\end{itemize}
	}
	
	\frame
	{
		\frametitle{Sch\'ema de construction du SOP}
		\begin{center}
			\includegraphics[width=\linewidth]{./figures/sop-definition.png}
		\end{center}		
	}

