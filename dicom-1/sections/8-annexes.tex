\section*{Annexes}

\frame
{
	\frametitle{UID}
	Unique Identifier
	\begin{itemize}
		\item 64 caract�res maximum, combinaison entre chiffres et points (OSI Object Identication, numeric form, ISO 8824)
		\item Afin d'en assurer l'unicit�, le UID est compos� par deux parties encha�n�es :
		\begin{itemize}
			\item Une racine li�e � une organisation (racine attribu�e par la norme ISO 9823-3)
			\item Un suffixe garanti unique par l'organisation
		\end{itemize}
	\end{itemize}
	La racine $1.2.840.10008$ identifie le standard DICOM
}

\frame
{
	\frametitle{�tiquettes d'identification}
	\begin{itemize}
		\item S�par�es en Groupe et �l�ment
		\item Notation hexad�cimale (0123456789ABDCEF)
		\item Identification unique, typage li�
		\begin{description}
			\item[(0010,0010)] Patient Name (PN)
			\item[(0010,0020)] Patient ID (LO)
			\item[(0008,0050)] Accession Number (SH)
			\item[(0020,000d)] Study Instance UID (UI)
			\item[(7fe0,0010)] Pixel Data (OW ou OB)
		\end{description}
	\end{itemize}
}

\frame
{
	\frametitle{Typage}
}

\frame
{
	\frametitle{SOP . Images}
	Selon la classe SOP, un objet DICOM peut d�crire une ou plusieurs images.
	
	Par exemple, il est normal de stocker une s�quence d'images pour une �chographie.
	
	G�n�ralement, on parle d'Enhanced DICOM pour les objets d�crivant plusieurs images.
}

