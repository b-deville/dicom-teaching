\section{Conclusions}

\frame
{
	\frametitle{Points forts}
	\begin{itemize}
		\item Standard complet et \'evolutif :
		\begin{itemize}
			\item Pas limit\'e \`a la radiologie (oncologie, h\'ematologie, dermatologie, ophtalmologie, cardiologie, ORL,\ldots).
			\item Annotations sur les images, compression.
		\end{itemize}
		\item Libert\'e du choix d'achat d'\'equipement car ind\'ependance du fournisseur.
		\item Distribution d'images possible : support des protocoles de communication standards, TCP/IP notamment.
	\end{itemize}
}

\frame
{
	\frametitle{Points faibles}
	\begin{itemize}
		\item Complexe :
		\begin{itemize}
			\item Difficile \`a comprendre.
			\item Informatique de niche.
		\end{itemize}
		\item N\'ecessit\'e de patience : int\'egration progressive par les fournisseurs.
		\item P\'erim\`etre limit\'e : traite de la connectivit\'e, mais pas des fonctionnalit\'es des logiciels.
		\item Besoin d'un niveau de conformit\'e.
	\end{itemize}
}

\frame
{
	\frametitle{DICOM Conformance}
	\begin{itemize}
		\item Le standard pr\'evoit un document "DICOM Conformance Statement" dont le plan et la structure sont pr\'ed\'efinis.
		\item Par ce document, le fournisseur pr\'ecise le niveau de conformit\'e de son \'equipement \`a la norme DICOM.
		\begin{itemize}
			\item Applicable sur chaque mod\`ele, chaque version.
			\item Le document suit un plan pr\'evu dans la norme.
			\item Liste des SOP Class support\'ees et des r\^oles assur\'es (SCU, SCP).
		\end{itemize}
	\end{itemize}
}

\frame
{
	\frametitle{Pi\`eges et difficult\'es}
	\begin{itemize}
		\item Divergences/erreurs d'interpr\'etation de la norme.
		\item DICOM autorise le stockage d'informations sp\'ecifiques.
		\begin{itemize}
			\item Risque: faire du propri\'etaire sous le label DICOM
		\end{itemize}
		\item DICOM propose diff\'erentes alternatives pour d\'ecrire l'information.
		\begin{itemize}
			\item Annotations: 3 moyens de les transmettre.
		\end{itemize}
		\item Manque d'information disponible \`a l'installation sur l'activation des services DICOM.
	\end{itemize}
}

\frame
{
	\frametitle{Quelques contre-v\'erit\'es}
	\begin{itemize}
		\item "Je n'arrive plus \`a envoyer mes images. C'est \`a cause de DICOM !"
		\begin{description}
			\item[Faux] DICOM utilise le r\'eseau, qui peut avoir ses d\'efaillances.
		\end{description}
		\item "DICOM d\'egrade la qualit\'e de mes images."
		\begin{description}
			\item[Faux] DICOM n'invente rien : il repose sur des formats d'image qui peuvent ou non \^etres compress\'es.
		\end{description}
		\item "Ne vous inqui\'etez pas, je supporte enti\`erement DICOM"
		\begin{description}
			\item[Faux] Remarque bien pr\'esomptueuse?
			Possible, mais est-ce r\'ealiste ?
		\end{description}
	\end{itemize}
}

\frame
{
	\frametitle{Synth\`ese}
	\begin{itemize}
		\item Norme incontournable.
		\item Tr\`es largement et rapidement adopt\'ee par la majorit\'e des acteurs.
		\item Plus large couverture que l'imagerie radiologique
		\begin{itemize}
			\item Rapport structur\'e.
			\item Ouverture \`a toutes les imageries.
		\end{itemize}
		\item Norme \'evolutive.
	\end{itemize}
}
