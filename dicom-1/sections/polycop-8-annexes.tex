\section*{Annexes}

\frame
{
	\frametitle{UID}
	Unique Identifier
	\begin{itemize}
		\item 64 caract\`eres maximum, combinaison entre chiffres et points (OSI Object Identication, numeric form, ISO 8824)
		\item Afin d'en assurer l'unicit\'e, le UID est compos\'e par deux parties encha\^in\'ees :
		\begin{itemize}
			\item Une racine li\'ee \`a une organisation (racine attribu\'ee par la norme ISO 9823-3)
			\item Un suffixe garanti unique par l'organisation
		\end{itemize}
	\end{itemize}
	La racine $1.2.840.10008$ identifie le standard DICOM
}

\frame
{
	\frametitle{\'Etiquettes d'identification}
	\begin{itemize}
		\item S\'epar\'ees en Groupe et \'el\'ement
		\item Notation hexad\`ecimale (0123456789ABDCEF)
		\item Identification unique, typage li\'e
		\begin{description}
			\item[(0010,0010)] Patient Name (PN)
			\item[(0010,0020)] Patient ID (LO)
			\item[(0008,0050)] Accession Number (SH)
			\item[(0020,000d)] Study Instance UID (UI)
			\item[(7fe0,0010)] Pixel Data (OW ou OB)
		\end{description}
	\end{itemize}
}

%\frame
%{
%	\frametitle{Typage}
%	Diff\'erents types possibles pour les Data Elements.
%	Analogie avec le jeu des formes pour les enfants.
%}

\frame
{
	\frametitle{SOP -- Images}
	Selon la classe SOP, un objet DICOM peut d\'ecrire une ou plusieurs images.
	
	Par exemple, il est normal de stocker une s\'equence d'images pour une \'echographie.
	
	G\'en\'eralement, on parle d'Enhanced DICOM pour les objets d\'ecrivant plusieurs images.
}

