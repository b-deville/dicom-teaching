\section{Fichier DICOM}

\frame
{
	\frametitle{Conventions DICOM}
	\begin{itemize}
		\item Single Frame
		\begin{itemize}
			\item<2-> Une image : stock\'ee dans un fichier.
			\item<3-> Une coupe = une image
		
			$\Rightarrow$ s\'erie de 100 coupes = 100 fichiers.
		\end{itemize}
		\item<4-> Multiframe
		\begin{itemize}
			\item<5-> Aussi appel\'e \emph{Enhanced DICOM}.
			\item<6-> Plusieurs images dans la m\^eme s\'equence.
			
			E.g. s\'equence vid\'eo d'\'echographie.
		\end{itemize}
		\item<7-> Arborescence des r\'epertoires/fichiers
		\begin{center}
			\includegraphics<8->[width=.8\linewidth]{./figures/arborescence.png}
		\end{center}

	\end{itemize}
}

\frame
{
	\frametitle{Fichier en pratique}
	
	\begin{itemize}
		\item Peut \^etre exp\'edi\'e par messagerie.
		\item<2-> Convertible en d'autres formats (JPEG, AVI, etc.)
		\item<3-> Extension : .dcm
		\item<4-> Exemple avec Horos.
	\end{itemize}
}

\frame
{
	\frametitle{Contenu d'un fichier .dcm}
	
	Un fichier DICOM est l'agr\'egation des \'el\'ements suivants :
	\begin{itemize}
		\item<2-> Pr\'e-ent\^ete :
		\begin{itemize}
			\item<3-> Pr\'eambule : 128 octets de donn\'ees "application".
			\item<4-> Pr\'efixe : 0x4449434D=DICM (4 octets).
		\end{itemize}
		\item<5-> Suite de Data Elements.
		En g\'en\'eral :
		\begin{itemize}
			\item<6-> Tag ;
			\item<7-> VR ;
			\item<8-> Taille ;
			\item<9-> et Valeur.
		\end{itemize}
		\item<10-> Et une image, qui est un Data Element (7FE0,0010).
	\end{itemize}
}


