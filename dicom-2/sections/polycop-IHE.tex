\section{DICOM et IHE}

 \frame
 {
 	\frametitle{Integrating the Healthcare Enterprise (IHE)}
	
	\begin{itemize} 	 	
		\item Am\'eliorer le partage d'informations m\'edicales entre logiciels, favoriser l'interop\'erabilit\'e.
		\item Spectre plus large que la seule imagerie.
		\item Promotion des standards d\'ej\`a \'etablis (\emph{e.g.} DICOM, HL7).
		\item Ensemble de profils d'int\'egration.
		\item Documentation compl\`ete sur le site Internet d'IHE (\url{https://www.ihe.net}).
		\item Des extensions nationales pour les particularismes locaux (\url{https://www.ihe-suisse.ch/fr}).
	\end{itemize}

 }

\frame
 {
 	\frametitle{Terminologie IHE}
	
	\begin{itemize}
		\item IHE d\'efinit les diff\'erents acteurs et leurs transactions.
		\item Comme DICOM, IHE indique les champs requis selon les cas, avec sa terminologie propre :	
		\begin{description}
			\item[R] Requis.
			\item[O] Optionnel.
		\end{description}
		\item Un champs R peut avoir des modificateurs (qui peuvent se combiner) :
		\begin{description}
			\item[R+] 	Exigence suppl\'ementaire par rapport \`a DICOM.
			\item[R*] Attribut requis, mais dont l'affichage n'est pas obligatoire.
		\end{description}
		\item Un attribut de type 3 dans un IOD, d\'efini comme R dans IHE, doit \^etre consid\'er\'e comme s'il \'etait de type 2.
	\end{itemize}
	
 }

\frame
 {
 	\frametitle{Exemple concret : partage d'images (XDS-I.b)}
	Revue de la Section 18 du profil IHE Radiology [RAD], portant sur le partage d'images entre entreprises.
 }
