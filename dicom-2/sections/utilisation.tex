\section{DICOM en pratique}

\frame
{
	\frametitle{Anonymisation}
	\begin{itemize}
		\item<1-> Utilisation d'images cliniques pour la recherche ou l'enseignement.
		\item<2-> Fichiers mis \`a disposition du public.
		\item<3-> N\'ecessit\'e d'anonymat : suppression des informations personnelles permettant d'identifier le patient.
		\begin{itemize}
			\item<4-> PatientsName (0010,0010)
			\item<5-> PatientID (0010,0020)
			\item<6-> PatientBirthDate (0010,0030)
			
			$\rightarrow$ de type 1 : \`a remplacer, pas supprimer !
			\item<7-> ReferringPhysicianName (0008,0090)
			\item<8-> etc.
			\item<9-> Potentiellement plus de 250 champs \`a supprimer ou vider !
		\end{itemize}
	\end{itemize}
}

\frame
{
	\frametitle{Achat d'un \'equipement}
	\begin{enumerate}
		\item<1-> Avant l'achat, soumission de l'appel d'offre :
		\begin{itemize}
			\item<2-> D\'efinition du sc\'enario de travail souhait\'e.
			
			Exemple : les images brutes export\'ees pourront \^etre r\'eutilis\'ees a posteriori.
			\item<3-> R\'edaction du cahier des charges DICOM.
			\begin{itemize}
				\item<4-> Pr\'eciser le niveau d'exigence de DICOM.
				
				$\rightarrow$ faire appel \`a un consultant ou \`a des coll\`egues,
				
				$\rightarrow$ ou acqu\'erir le savoir-faire en interne.
				\item<5-> Demander le Document de Conformit\'e DICOM (DICOM Conformance Statement).
			\end{itemize}
		\end{itemize}
		\item<6-> Acceptation protocol\'ee.
		\begin{itemize}
			\item<7-> V\'erification de DICOM.
			\item<8-> V\'erification du/des sc\'enario/ii requis.
			\item<9-> Tests.
		\end{itemize}
	\end{enumerate}
}

\frame
{
	\frametitle{DICOM Conformance}
	Point faible abord\'e rapidement la derni\`ere fois.
	\begin{itemize}
		\item<2-> Le standard pr\'evoit un document "DICOM Conformance Statement" dont le plan et la structure sont pr\'ed\'efinis.
		\item<3-> Par ce document, le fournisseur pr\'ecise le niveau de conformit\'e de son \'equipement au standard DICOM.
		\begin{itemize}
			\item<4-> Applicable sur chaque mod\`ele, chaque version.
			\item<5-> Le document suit un plan pr\'evu dans le standard.
			\item<6-> Liste des SOP Class support\'ees et des r\^oles assur\'es (SCU, SCP).
		\end{itemize}
	\end{itemize}
}

\frame
{
	\frametitle{Exemple de DICOM Conformance Statement}
}

\frame
{
	\frametitle{Services \`a demander}
	Exemples typiques de services DICOM \`a exiger pour un scanner :
	\begin{itemize}
		\item<2-> Worklist (SCU) : Import de la liste de patients.
		\item<3-> Store : envoi des images par r\'eseau
		\begin{itemize}
			\item<4-> Envoi : modalit\'es (SCU) : \textbf{CT}.
			\item<5-> R\'eception (SCP) : \textbf{CT}, \textbf{IRM}.
		\end{itemize}
		\item<6-> Print (SCU) : envoi des images pour impression
	\end{itemize}
	
	\begin{center}
		\includegraphics<7->[width=\linewidth]{./figures/services-ct.png}
	\end{center}

}

\frame
{
	\frametitle{\'Equipements non standards}
	
	\begin{itemize}
		\item Int\'egrer dans un workflow DICOM : installer une passerelles de conversion.
		
		\begin{center}
			\includegraphics<2->[width=\linewidth]{./figures/passerelle.png}
		\end{center}
		\item<3-> Limitation : images stock\'ees en mode Secondary Capture (IOD le plus simple de DICOM), les donn\'ees d'acquisition des images sont perdues.
	\end{itemize}
}

