\frame
{
	\frametitle{Et la s\'ecurit\'e dans tout \c ca?}
	\begin{itemize}
		\item DICOM n'a pas \'et\'e con\c cu au d\'epart avec le souci de la s\'ecurit\'e des donn\'ees.
		\item La RGPD est encore r\'ecente.
		\item Les attaques se multiplient et les donn\'ees m\'edicales sont pr\'ecieuses.
		\item Certaines donn\'ees patient sont dans les images.
	\end{itemize}
	Il y a donc des \'evolutions.
	\begin{itemize}
		\item<2-> DICOMweb\texttrademark: profiter de la s\'ecurisation des protocoles Internet pour migrer les services DICOM.
		\item<3-> Outils de d\'esidentification (e,g, Karnak).
	\end{itemize}
}

% \frame
% {
% 	\frametitle{D\'esidentification}
% 	\begin{itemize}
% 		\item Utilisation d'images cliniques pour la recherche ou l'enseignement.
% 		\item<2-> Fichiers mis \`a disposition du public.
% 		\item<3-> N\'ecessit\'e d'anonymat : suppression des informations personnelles permettant d'identifier le patient.
% 		\begin{itemize}
% 			\item<4-> PatientsName (0010,0010)
% 			\item<4-> PatientID (0010,0020)
% 			\item<4-> PatientBirthDate (0010,0030)
			
% 			$\rightarrow$ de type 1 : \`a remplacer, pas supprimer !
% 			\item<4-> ReferringPhysicianName (0008,0090)
% 			\item<4-> etc.
% 			\item<4-> Potentiellement plus de 250 champs \`a supprimer ou \`a vider !
% 		\end{itemize}
% 	\end{itemize}
% }